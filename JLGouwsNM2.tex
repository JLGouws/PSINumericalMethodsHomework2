\documentclass[12pt,a4]{article}
\usepackage{physics, amsmath,amsfonts,amsthm,amssymb, mathtools,steinmetz, gensymb, siunitx}	% LOADS USEFUL MATH STUFF
\usepackage{xcolor,graphicx}
\usepackage{caption}
\usepackage{subcaption}
\usepackage[left=45pt, top=20pt, right=45pt, bottom=45pt ,a4paper]{geometry} 				% ADJUSTS PAGE
\usepackage{setspace}
\usepackage{tikz}
\usepackage{pgf,tikz,pgfplots,wrapfig}
\usepackage{mathrsfs}
\usepackage{fancyhdr}
\usepackage{float}
\usepackage{array}
\usepackage{booktabs,multirow}
\usepackage{bm}
\usepackage{tensor}
%\usepackage{listings}
\usepackage{slashed}
\usepackage{tikz-feynman}
% \lstset{
%    basicstyle=\ttfamily\small,
%    numberstyle=\footnotesize,
%    numbers=left,
%    backgroundcolor=\color{gray!10},
%    frame=single,
%    tabsize=2,
%    rulecolor=\color{black!30},
%    title=\lstname,
%    escapeinside={\%*}{*)},
%    breaklines=true,
%    breakatwhitespace=true,
%    framextopmargin=2pt,
%    framexbottommargin=2pt,
%    inputencoding=utf8,
%    extendedchars=true,
%    literate={á}{{$\rho$}}1 {ã}{{\~a}}1 {é}{{\'e}}1,
%}
\DeclareMathOperator{\sign}{sgn}

\usetikzlibrary{decorations.text, calc}
\pgfplotsset{compat=1.7}

\usetikzlibrary{decorations.pathreplacing,decorations.markings}
\usepgfplotslibrary{fillbetween}

\newcommand{\vect}[1]{\boldsymbol{#1}}

\usepackage{hyperref}

%\usepackage[style= ACM-Reference-Format, maxbibnames=6, minnames=1,maxnames = 1]{biblatex}
%\addbibresource{references.bib}


\hypersetup{pdfborder={0 0 0},colorlinks=true,linkcolor=black,urlcolor=cyan,}
\allowdisplaybreaks
%\hypersetup{
%
%    colorlinks=true,
%
%    linkcolor=blue,
%
%    filecolor=magenta,      
%
%    urlcolor=cyan,
%
%    pdftitle={An Example},
%
%    pdfpagemode=FullScreen,
%
%    }
%}

\title{
\textsc{Numerical Methods Homework 2}
}
\author{\textsc{J L Gouws}
}
\date{\today
\\[1cm]}



\usepackage{graphicx}
\usepackage{array}
\usepackage{jlcode}



\begin{document}
\thispagestyle{empty}

\maketitle

\begin{enumerate}
  \item
    The source code is available \href{https://github.com/JLGouws/PSINumericalMethodsHomework2}{here}. I did it in a script as it was easier for me to work with.
    I used the following packages: 
Printf,
LaTeXStrings,
CairoMakie, 
HCubature, 
SphericalHarmonics, 
LegendrePolynomials, and
DifferentialEquations.

    \begin{enumerate}
      \item
        I use the spherical harmonics package to determine the spherical harmonics and then hcubature to determine the coeficients of the function expansion.
\begin{jllisting}
expandSphericalHarmonics2(f, lₘₐₓ) = hcubature(
  x -> sin(x[1]) * f(x[1], x[2]) .* conj(flattenSHArray(computeYlm(x[1], x[2], lmax = lₘₐₓ)))
  , [0, 0], [π, 2 * π]
)
\end{jllisting}
  Where I wrote a function to convert the output of spherical harmonic function into a flat array:
\begin{jllisting}
function flattenSHArray(shArray, lₘₐₓ)
  l = 0
  m = 0
  coeffVector::Array{ComplexF64} = []
  for a in 0:lₘₐₓ * (lₘₐₓ + 2)
    push!(coeffVector, shArray[(l, m)])
    if m == l
      l+=1
      m = -l
    else
      m+=1
    end
  end
  coeffVector
end
\end{jllisting}
\item
  I will use these initial conditions for the wave equation:

\begin{jllisting}
  ϕ₀(θ, ϕ) = exp( - θ^2 / 0.4);
  ψ₀(θ, ϕ) = 0;
\end{jllisting}
And next I will detail the more technical details of the numerical solver.
I wrote this function to differentiate an array of spherical harmonic coefficents from an expansion.
\begin{jllisting}
function LaplacianSphericalCoefficientsArrays(coeffs)
  l = 0
  m = 0
  coeffVector::Array{ComplexF64} = []
  for a in coeffs
    push!(coeffVector, - l * (l + 1) * a) #multiply coefficient by correct factor
    if m == l
      l+=1
      m = -l
    else
      m+=1
    end
  end
  coeffVector
end
\end{jllisting}
        Now we have all the ingredients to write a function that will solve the wave equation:
\begin{jllisting}
function waveSystem(U)
    half = Int(size(U)[1] // 2)
    ϕ = U[begin:half] #split the state vector in two
    ψ = U[half + 1:end]    
    vcat(ψ, LaplacianSphericalCoefficientsArrays(ϕ)) #concatenate the two
end

function solveWaveEquation(ϕ₀, ψ₀, N, t₀, t₁)
  ϕ₀SphericalCoeffs = expandSphericalHarmonics2(ϕ₀, N)[1]; #get expanstion of
  ψ₀SphericalCoeffs = expandSphericalHarmonics2(ψ₀, N)[1]; # initial conditions

  alg = DP5()
  U0 = vcat( ϕ₀SphericalCoeffs, ψ₀SphericalCoeffs)         #I solve this with an
  prob = ODEProblem((U,p,t) -> waveSystem(U), U0, (t₀, t₁))# an integrator from a pacakage 
  solve(prob, alg)
end
\end{jllisting}
As can be seen from the above code I use a solver in the DifferentialEquations package.
And do some accounting to keep track of the different arrays for $\phi$ and $\psi$.
        And lastly we need a function that will give us a function from these coefficients: 
\begin{jllisting}
function functionFromSphericalCoefficients(coeffs)
  l = 0
  m = 0
  lₘₐₓ = 0
  for a in 1:size(coeffs)[1]
    lₘₐₓ = l
    if m == l
      l+=1
      m = -l
    else
      m+=1
    end
  end
  function f(θ, ϕ)
    harmonics = flattenSHArray(computeYlm(θ, ϕ, lmax = lₘₐₓ))
    harmonics' * coeffs
  end
end
\end{jllisting}
      \item
        This part is now easy since we have all the ingredients.
\begin{jllisting}
  ϕ₀(θ, ϕ) = exp( - θ^2 / 0.4);
  ψ₀(θ, ϕ) = 0;
  sol = solveWaveEquation(ϕ₀, ψ₀, lmax, 0, 10)
\end{jllisting}

And I use this solution to reconstruct the function for the solution.
\begin{jllisting}
sizA = Int(size(sol.u[1])[1] / 2)

firstprofileWhole = sol[1][begin:sizA]
myFuncFirst = functionFromSphericalCoefficients(firstprofileWhole)
\end{jllisting}

      \item
        Below are figures of the solution at different times. The plots show the solutions of the wave equations as function of $\theta$ for the $\phi = \pi / 2$.
        I found these graphs the easiest to interpret.
        I might have gone overboard with the testing--I did five different solutions for cut-offs $l_\text{max} = 2, 4, 6, 8, 10$.
        As Figure~\ref{fig:lmax2} shows, in particular the first subfigure, for $l_\text{max} = 2$, the expansion does not capture the profile of the initial conditions very well, but the problem can still be evolved and a solution obtained..
        As Figure~\ref{fig:lmax4} shows, for $l_\text{max} = 4$, the motion of the wave is clearer, but there are some ripples in the solution that are not apparent in the true solution. 
        The ripples present for the cut-off $l_\text{max} = 4$ slowly become smaller for larger cut-offs, which is indicative of a convergent solution.
        This covergence can be qualititatively seen by comparing Figures~\ref{fig:lmax6}, \ref{fig:lmax8} and \ref{fig:lmax10}, keeping in mind that the profiles are printed for slightly different times.

        \begin{figure}[H]
          \centering
          \includegraphics[scale = 0.23]{GaussianWaveSurfaceFromScriptTimesSlice2.png}
          \caption{Solution of the wave equation with cut-off $l_\text{max} = 2$}
          \label{fig:lmax2}
        \end{figure}
        \newpage
        \begin{figure}[H]
          \centering
          \includegraphics[scale = 0.23]{GaussianWaveSurfaceFromScriptTimesSlice4.png}
          \caption{Solution of the wave equation with cut-off $l_\text{max} = 4$}
          \label{fig:lmax4}
        \end{figure}
        \newpage
        \begin{figure}[H]
          \centering
          \includegraphics[scale = 0.23]{GaussianWaveSurfaceFromScriptTimesSlice6.png}
          \caption{Solution of the wave equation with cut-off $l_\text{max} = 6$}
          \label{fig:lmax6}
        \end{figure}
        \newpage
        \begin{figure}[H]
          \centering
          \includegraphics[scale = 0.23]{GaussianWaveSurfaceFromScriptTimesSlice8.png}
          \caption{Solution of the wave equation with cut-off $l_\text{max} = 8$}
          \label{fig:lmax8}
        \end{figure}
        \newpage
        \begin{figure}[H]
          \centering
          \includegraphics[scale = 0.23]{GaussianWaveSurfaceFromScriptTimesSlice10.png}
          \caption{Solution of the wave equation with cut-off $l_\text{max} = 10$}
          \label{fig:lmax10}
        \end{figure}
        \newpage
    \end{enumerate}
    \newpage
  \item
    I also did the heat equation out of interest.
\begin{jllisting}
function heat(U)
  LaplacianSphericalCoefficientsArrays(U)
end

ϕ₀(θ, ϕ) = exp(- θ^2)
U0 = flattenSHArray(expandSphericalHarmonics2(ϕ₀, 15)[1])
prob = ODEProblem((U,p,t) -> heat(U), U0, (0.0, 10.0))

alg = Rodas5(autodiff = false)
solheat = solve(prob, alg);
\end{jllisting}
  
  Figure~\ref{fig:heat} shows the numerical solution of the heat equation with the heat dispersing over the sphere.
  \begin{figure}[H]
    \centering
    \includegraphics[scale = 0.23]{HeatTimesOnSphere.png}
    \caption{Solution of the heat equation with cut-off $l_\text{max} = 15$}
    \label{fig:heat}
  \end{figure}
\end{enumerate}
\end{document}
