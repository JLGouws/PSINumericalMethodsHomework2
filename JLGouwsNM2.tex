\documentclass[12pt,a4]{article}
\usepackage{physics, amsmath,amsfonts,amsthm,amssymb, mathtools,steinmetz, gensymb, siunitx}	% LOADS USEFUL MATH STUFF
\usepackage{xcolor,graphicx}
\usepackage{caption}
\usepackage{subcaption}
\usepackage[left=45pt, top=20pt, right=45pt, bottom=45pt ,a4paper]{geometry} 				% ADJUSTS PAGE
\usepackage{setspace}
\usepackage{tikz}
\usepackage{pgf,tikz,pgfplots,wrapfig}
\usepackage{mathrsfs}
\usepackage{fancyhdr}
\usepackage{float}
\usepackage{array}
\usepackage{booktabs,multirow}
\usepackage{bm}
\usepackage{tensor}
\usepackage{listings}
\usepackage{slashed}
\usepackage{tikz-feynman}
 \lstset{
    basicstyle=\ttfamily\small,
    numberstyle=\footnotesize,
    numbers=left,
    backgroundcolor=\color{gray!10},
    frame=single,
    tabsize=2,
    rulecolor=\color{black!30},
    title=\lstname,
    escapeinside={\%*}{*)},
    breaklines=true,
    breakatwhitespace=true,
    framextopmargin=2pt,
    framexbottommargin=2pt,
    inputencoding=utf8,
    extendedchars=true,
    literate={á}{{$\rho$}}1 {ã}{{\~a}}1 {é}{{\'e}}1,
}
\DeclareMathOperator{\sign}{sgn}

\usetikzlibrary{decorations.text, calc}
\pgfplotsset{compat=1.7}

\usetikzlibrary{decorations.pathreplacing,decorations.markings}
\usepgfplotslibrary{fillbetween}

\newcommand{\vect}[1]{\boldsymbol{#1}}

\usepackage{hyperref}

%\usepackage[style= ACM-Reference-Format, maxbibnames=6, minnames=1,maxnames = 1]{biblatex}
%\addbibresource{references.bib}


\hypersetup{pdfborder={0 0 0},colorlinks=true,linkcolor=black,urlcolor=cyan,}
\allowdisplaybreaks
%\hypersetup{
%
%    colorlinks=true,
%
%    linkcolor=blue,
%
%    filecolor=magenta,      
%
%    urlcolor=cyan,
%
%    pdftitle={An Example},
%
%    pdfpagemode=FullScreen,
%
%    }
%}

\title{
\textsc{Standard Model Homework 2}
}
\author{\textsc{J L Gouws}
}
\date{\today
\\[1cm]}



\usepackage{graphicx}
\usepackage{array}




\begin{document}
\thispagestyle{empty}

\maketitle

This homework is a bit rough, but I am submitting it anyway for feedback.
\begin{enumerate}
  \item
    \begin{enumerate}
      \item
      \item
        Below are figures of the solution at different times:
        \begin{figure}[!ht]
          \centering
          \includegraphics[scale = 0.23]{GaussianWaveSurfaceFromScriptTimesSlice2.png}

          \caption{$l_\text{max} = 2$}
        \end{figure}
        \newpage
        \begin{figure}[!ht]
          \centering
          \includegraphics[scale = 0.23]{GaussianWaveSurfaceFromScriptTimesSlice4.png}
          \caption{$l_\text{max} = 4$}
        \end{figure}
        \newpage
        \begin{figure}[!ht]
          \centering
          \includegraphics[scale = 0.23]{GaussianWaveSurfaceFromScriptTimesSlice6.png}
          \caption{$l_\text{max} = 6$}
        \end{figure}
        \newpage
        \begin{figure}[!ht]
          \centering
          \includegraphics[scale = 0.23]{GaussianWaveSurfaceFromScriptTimesSlice8.png}
          \caption{$l_\text{max} = 8$}
        \end{figure}
        \newpage
        \begin{figure}[!ht]
          \centering
          \includegraphics[scale = 0.23]{GaussianWaveSurfaceFromScriptTimesSlice10.png}
          \caption{$l_\text{max} = 10$}
        \end{figure}
        \newpage
    \end{enumerate}
\end{enumerate}
\end{document}
